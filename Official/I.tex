\begin{problem}{出给 paul-lu 的数数题}{standard input}{standard output}{6 seconds}{128 megabytes}

    中国科学院大学的著名 JB 选手 bibibibi,想当年还在北理工打算法竞赛时候就是一个全才,号称北京全能王,所有的算法题目里就没有他不能嘴巴 AC 的。这天他听说 Paul(lu) Lu(pao)学长挺会数数的,于是决定出一道数数的题目来考一考他:

    已知在一个 $n\times n$ 的棋盘里,每个格子都可以填上一个范围为 $[1,k]$ 的正整数。定义棋盘中的某个格点是 \textit{bi点}\ 当且仅当满足:

    $\bullet$ 该格点的整数值\textbf{严格大于}本行其他所有格点的整数值。

    $\bullet$ 该格点的整数值\textbf{严格大于}本列其他所有格点的整数值。
    
    设 $B_i$ 为棋盘中恰好存在 $i$ 个 \textit{bi点}\ 的方案数,请你计算 $\sum\limits_{i=0}^{n^2} (i^2\cdot B_i) $,答案可能很大,请取模 $998244353$。

    \InputFile
    
    第一行输入一个正整数 $T\ (1\le T\le 20)$,表示数据组数。

    接下来 $T$ 组数据,每组数据输入两个正整数 $n$ 和 $k\ (1\le n, k\le 200)$,由空格间隔开,分别表示棋盘的大小和每个格点内可填数的范围上限。
    
    \OutputFile
    
    对于每组数据,请输出一行,表示计算结果取模 $998244353$ 的答案,注意换行。
    
    \Example
    
    \begin{example}
    \exmp{
        3
        1 2
        2 2
        3 2
    }{
        2
        12
        216
    }%
    \end{example}
    

\end{problem}