\begin{problem}{多项式求导}{standard input}{standard output}{1 second}{128 megabytes}

    一元多项式形如 $a_nx^n+a_{n-1}x^{n-1}+\cdots+a_2x^2+a_1x+a_0​$,它是代数学研究的基本对象之一。求导是数学计算中一种常用的计算方法,它的表示当自变量的增量趋于零时因变量的增量与自变量的增量之商的极限。

    早在高中学习导数的时候,天天就觉得给一元多次多项式求导是个简单到浪费脑细胞的工作。于是他在课间使用 C 语言写了一个给多项式求导的小程序。时隔这么多年,他早已找不到当年写的这段小程序代码了,希望聪明的你可以帮他重新完成这段给一元多项式求导的程序代码。

    \InputFile
    
    输入共两行,第一行输入两个正整数 $n$ 和 $k\ (1\le n,k\le 100)$,分别表示输入的多项式次数 $n$,和程序需要对该多项式进行求导运算的阶数 $k$。

    第二行输入 $n+1$ 个非负整数 $a_n,\cdots ,a_1,a_0$ 由空格间隔开,其中第 $n-i+1$ 个整数为 $a_i\ (0\le a_i\le 100)$,描述输入多项式第 $i$ 次项 $x^{i}$ 的系数。
    
    \OutputFile
    
    请输出一行,包含 $n+1$ 个非负整数 $a_n,\cdots ,a_1,a_0$ 由空格间隔开,其中第 $n-i+1$ 个整数为 $a_i$,描述输入多项式求导后第 $i$ 次项 $x^i$ 的系数。由于系数可能很大,对于每个 $a_i$ 你只需要输出它取模 $2019$ 之后的答案。
    
    \Example
    
    \begin{example}
    \exmp{
        5 1
        1 2 3 4 5 6

    }{
        0 5 8 9 8 5
    }%
    \exmp{
        5 2
        1 2 3 4 5 6
    }{
        0 0 20 24 18 8
    }%
    \end{example}

    \Explanation

    样例一和样例二的多项式为 $f(x)=x^5+2x^4+3x^3+4x^2+5x^1+6$,它的一阶导数为 $f'(x)=5x^4+8x^3+9x^2+8x^1+5$,它的二阶导数为 $f''(x)=20x^3+24x^2+18x+8$。

\end{problem}