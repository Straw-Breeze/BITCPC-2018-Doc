\begin{problem}{神圣的 F2 连接着我们}{standard input}{standard output}{2 seconds}{256 megabytes}

    小白非常喜欢玩“县际争霸”这款游戏,虽然他的技术并不容乐观。“县际争霸”的地图共有两个县,每个县里各有 $n$ 个据点。同一个县之间的据点是互不连通的,两个县之间的据点也是互不连通的。小白的 $p$ 个战斗单位在第一个县的第 $x_1,x_2,\cdots,x_p$ 个据点中,而对手的 $q$ 个建筑单位在第二个县第 $y_1,y_2,\cdots,y_q$ 个据点中。

    为了发起进攻,小白建造了很多的“折跃棱镜”。折跃棱镜可以帮助小白的单位在两个县之间移动,而且可以有多个单位同时通过折跃棱镜。具体地说,一个折跃棱镜包含有 $5$ 个参数信息 $a$、$b$、$c$、$d$ 和 $w$,代表位于第一个县任意第 $x\ (a\le x\le b)$ 个据点的战斗单位可以花费 $w$ 单位时间到达第二个县的任意第 $y\ (c\le y\le d)$ 个据点。折跃棱镜的通道是双向的,所以位于第二个县任意第 $y\ (c\le y\le d)$ 个据点的战斗单位也可以花费 $w$ 单位时间到达第一个县的任意第 $x\ (a\le x\le b)$ 个据点。

    如果一个小白的战斗单位到达了一个有敌方建筑单位的据点,那么这个战斗单位就会即刻投入战斗。当小白所有的战斗单位都投入了战斗之后,对手会感觉到压力太大而主动投降。小白想尽快结束这场战斗,请聪明的你帮他算一算,如果采用最优的调度策略,对手最早将在什么时刻投降(假设当前局面是零时刻)。如果存在战斗单位始终无法投入战斗,则请输出\texttt{boring game}。

    \InputFile
    
    输入共 $m+3$ 行,第一行输入四个正整数 $n$、$m$、$p$ 和 $q\ (1\le n,m,p,q\le 10^5)$ 由空格间隔开,分别表示每个县的据点数量、折跃棱镜数量、小白的战斗单位数量和对手的建筑单位数量。

    接下来的 $m$ 行中,第 $i$ 行输入五个正整数 $a_i$、$b_i$、$c_i$、$d_i$ 和 $w\ (1\le a_i\le b_i\le n,1\le c_i\le d_i\le n,1\le w_i\le 10^9)$ 由空格间隔开,表示第 $i$ 个折跃棱镜的参数,具体含义见题目描述。

    接下来一行输入 $p$ 个正整数 $x_1,x_2,\cdots,x_p\ (1\le x_i\le n)$ 由空格间隔开,分别表示小白的战斗单位所在的据点位置。

    最后一行输入 $q$ 个正整数 $y_1,y_2,\cdots,y_q\ (1\le y_i\le n)$ 由空格间隔开,分别表示对手的建筑单位所在的据点。
    
    \OutputFile
    
    如果对手最终会主动投降,则请输出一个非负整数,表示在小白的最优调度策略下对手最早的投降时间;如果存在战斗单位始终无法投入战斗,则请输出\texttt{boring game}。
    
    \Example
    
    \begin{example}
    \exmp{
        5 3 2 2
        2 4 1 3 1
        1 1 4 5 3
        1 2 3 4 2
        2 3
        4 5

    }{
        4
    }%
    \end{example}

    \Note
    
    样例中,一种最优调度策略是:第一个战斗单位从第一县的 $2$ 号据点折跃到第二县的 $4$ 号据点即可投入战斗,这将花费 $2$ 单位时间。同时第二个战斗单位从第一县的 $3$ 号据点折跃到第二县的 $3$ 号据点再折跃到第一县的 $2$ 号据点最后折跃到第二县的 $4$ 号据点也可投入战斗,这将花费 $4$ 单位时间。

\end{problem}