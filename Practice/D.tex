\begin{problem}{最小值}{standard input}{standard output}{2 seconds}{128 megabytes}

    欢迎来到\textit{北京理工大学第十四届“连山科技”程序设计大赛}。命题人秉承着 “让题目在题面友好的基础上有一点点思维难度” 的出题原则,为大家安排了一些好玩又有趣的算法题。
    
    为了向大家展示命题人的友好,咱们接着来做一道机智题:

    对于给定的数列 $a_1,a_2,\cdots,a_n$,每次可以选择一个编号 $i\ (1<i<n)$,将 $a_i$ 变成 $a_{i+1}+a_{i−1}−a_i$。请问在经过任意多次的操作后,该数列的数字总和最小为多少?

    \InputFile

    第一行一个正整数 $n\ (1\le n\le 10^5)$,描述数列长度。

    第二行输入 $n$ 个非负整数 $a_1,a_2,\cdots, a_n\ (0\le a_i\le 1000)$ 由空格间隔开,描述数列的元素。
    
    \OutputFile
    
    请输出一个整数,表示经过任意多次的操作后,该数列的数字总和的最小值。
    
    \Example
    
    \begin{example}
    \exmp{
        8
        2 0 1 9 0 4 1 3
    }{
        -49
    }%
    \end{example}

\end{problem}